\documentclass[11pt]{article}
\usepackage{amsmath,amssymb,amsfonts,epsfig,algorithm,algorithmic,url}

\textheight 8.8truein
\parskip 0.1in
\topmargin -0.5truein
\textwidth 6.5truein
\oddsidemargin -0.05in
\evensidemargin -0.05in
% \renewcommand{\baselinestretch}{1.2}   %line space adjusted here
\setcounter{footnote}{0}
\sloppy

\renewcommand{\theequation}{\thesection.\arabic{equation}}
\newcommand{\newsection}{\setcounter{equation}{0}\section}

% redefining commonly used symbols
\DeclareMathOperator{\trace}{Tr}

\newtheorem{theorem}{Theorem}
\newtheorem{proposition}[theorem]{Proposition}
\newtheorem{lemma}[theorem]{Lemma}
\newtheorem{corollary}[theorem]{Corollary}
\newtheorem{definition}[theorem]{Definition}
\newtheorem{example}[theorem]{Example}

\begin{document}
\title{\bf Sampling Methods for Optimal Control}
\date{\today}
\author{Bolei Di}
%\normalsize{}
\maketitle
\begin{abstract}

\end{abstract}

\section{Basic Theory}\label{sec:intro}
This is the theorem that connects sampling to optimization.
\begin{gather*}
\min_x f(x) = \min_p E[f(x)]
\end{gather*}
Suppose that $f(x)$ reaches its minimum at $x^{\star}$, then one optimal distribution over $x$ is $p^{\star}(x)=\delta(x-x^{\star})$. 
Here we relax the stochastic optimization on the right side by adding a term that is the Kullback-Leibler divergence of $p(x)$ and another arbitrary distribution $q(x)$:
\begin{gather*}
\min_x f(x) = \min_p E[f(x)] \leq \min_p E[f(x)] + \lambda D_{KL}(p||q)
\end{gather*}
where $D_{KL}(p||q)=\int_x p(x) \ln \frac{p(x)}{q(x)} \text{d}x \geq 0$, which is always non-negative. 
The optimal solution to the relaxed problem has a simple, analytic form, surprisingly:
\begin{gather*}
p^{\star}(x) = \frac{1}{Z} \exp{-\frac{1}{\lambda}f(x)} q(x)
\end{gather*}
where $Z$ is a normalizing factor. 
We can choose the arbitrary distribution $q(x)$ to be Gaussian and define $L(x)=\exp{-\frac{1}{\lambda}f(x)}$, we can get:
\begin{gather*}
p^{\star}(x) = Z^{-1} L(x) \mathcal{N}(x)
\end{gather*}
For an optimal control problem, we can define the cost function $f(x)$ to be minimized in terms of the states and inputs $x$, hence we often have an analytic expression for $L(x)$. Then using the elliptical slice sampling technique we can generate random numbers following $p^{\star}(x)$ distribution. 


\end{document}
